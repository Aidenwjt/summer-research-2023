\documentclass[11pt]{article}
\setlength{\textwidth}{430pt}\setlength{\oddsidemargin}{11pt}
\usepackage{amssymb}
\usepackage{amsthm}
\usepackage{amsmath}
\usepackage{enumerate}
\usepackage{fancyhdr}
\usepackage[bottom=0.9in,top=0.9in]{geometry}

\begin{document}
\pagestyle{fancy}
\fancyhead{}
\fancyhead[L]{\textbf{\rightmark}}
\fancyhead[C]{\textbf{Summer Studentship Proposal 2023}}

\begin{enumerate}

\item Title - 10 Words Max:
\begin{itemize}
\item A Study of Apaptive Finite Element Mehtods, and their Applications.
\item A Study of Adaptive Finite Element Methods, and its Applications.
\item A Study of Functional Spaces, and Adaptive Finite Element Methods.
\end{itemize}

\item Overview/Abstract - 300 Words Max: \\
% Find better definition
By definition, Partial Differential Equations (PDE) are equations which impose relationships
between the partial derivatives of a multi-variable function, where a partial derivative is
essentially the rate of change of a quantity while holding the several other interdependent quantities constant.
PDEs find their merit in that many physical, biological, and engineering related problems can be represented by means of PDEs.
However, these equations often do not have exact solutions, so it becomes important to have Numerical Methods to approximate these solutions.
Adaptive Finite Element Methods (AFEM)
are fundamental Numerical Methods used to approximate PDEs, where we implement these methods by means of computers, specifically, by means
of algorithms that use simple arithmetic operations that result in approximate solutions in a numerical form.
With these definitions in mind, the goals of this project are to first conduct a study into Functional Spaces such as
Hilbert and Sobolev Spaces, which are relevant to the analysis of PDEs and the construction of their solutions.
Then, we will progress into the study of Numerical Methods induced by weak formulations
of PDEs, namely, Finite Element Methods. And finally, we will study state of the art AFEMs
by researching and applying relevant Numerical Methods from a survey in this topic.

\item Originality, Creativity, and Significance - 300 Words Max: \\
The originality of this project will come from our research into these state of the art AFEMs, and how we can apply these methods to model different problems,
primarily, in the form of computer programs. The creativity of this project will come from how we decide to implement these programs, as there
are many different ways to implement the same program or algorithm, which directly represent the creativity and personality of the programmer.
Also, creativity will come from how we decide to optimize and refine these programs or algorithms, as a programmer has to be very clever in how they
design their programs, and also in how they continue to improve them in the aim to make them more efficient, stable, and well-conditioned.
Finally, the significance of this project will come from the study and analysis of PDEs, and the theory behind these equations. As stated before, many physical, biological, and engineering related
problems can be represented mathematically by means of PDEs. Specifically, we find applications in many important fields of study such as Schrödinger's equation in Physics,
Reaction-Diffusion equations in Biology and Chemistry, the Black-Scholes equation in Economics, etc... It is easy to see that PDEs are used in basically all scientific areas,
but come with the caveat that many do not have exact solutions, which is where Numerical Methods, such as AFEMs, gain their significance.


\item Potential Benefit - 250 Words Max: \\
Some text

\item Relevant Experience - 250 Words Max: \\
Some text

\end{enumerate}

\end{document}
