\documentclass[11pt]{article}
\setlength{\textwidth}{430pt}\setlength{\oddsidemargin}{11pt}
\usepackage{amssymb}
\usepackage{amsthm}
\usepackage{amsmath}
\usepackage{enumerate}
\usepackage{fancyhdr}
\usepackage[bottom=0.9in,top=0.9in]{geometry}

\begin{document}
\pagestyle{fancy}
\fancyhead{}
\fancyhead[L]{\textbf{\rightmark}}
\fancyhead[C]{\textbf{Summer Studentship Proposal 2023}}

\begin{enumerate}

\item Title - 10 Words Max:
\begin{itemize}
\item A Study of Apaptive Finite Element Mehtods, and their Applications.
\item A Study of Adaptive Finite Element Methods, and its Applications.
\item A Study of Functional Spaces, and Adaptive Finite Element Methods.
\end{itemize}

\item Overview/Abstract - 300 Words Max: \\
% Find better definition
By definition, Partial Differential Equations (PDE) are equations which impose relationships
between the partial derivatives of a multi-variable function, where a partial derivative is
essentially the rate of change of a quantity while holding the several other interdependent quantities constant.
PDEs find their merit in that many physical, biological, and engineering related problems can be represented by means of PDEs.
However, these equations often do not have analytic/exact solutions, so it becomes important to have Numerical Methods to approximate these solutions.
Adaptive Finite Element Methods (AFEM)
are fundamental Numerical Methods used to approximate PDEs, where we implement these methods by means of computers, specifically, by means
of algorithms that use simple arithmetic operations that result in approximate solutions in a numerical form.
With these definitions in mind, the goals of this project are to first conduct a study into Functional Spaces such as
Hilbert and Sobolev Spaces, which are relevant to the analysis of PDEs and the construction of their solutions.
Then, we will progress into the study of Numerical Methods induced by weak formulations
of PDEs, namely, Finite Element Methods. And finally, we will study state of the art AFEMs
by researching and applying relevant Numerical Methods from a survey in this topic.

\item Originality, Creativity, and Significance - 300 Words Max: \\
Some text

\item Potential Benefit - 250 Words Max: \\
Some text

\item Relevant Experience - 250 Words Max: \\
Some text

\end{enumerate}

\end{document}
