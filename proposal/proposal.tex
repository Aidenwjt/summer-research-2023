\documentclass[11pt]{article}
\setlength{\textwidth}{430pt}\setlength{\oddsidemargin}{11pt}
\usepackage{amssymb}
\usepackage{amsthm}
\usepackage{amsmath}
\usepackage{enumerate}
\usepackage{fancyhdr}
\usepackage[bottom=0.9in,top=0.9in]{geometry}

\begin{document}
\pagestyle{fancy}
\fancyhead{}
\fancyhead[L]{\textbf{\rightmark}}
\fancyhead[C]{\textbf{Summer Studentship Proposal 2023}}

\begin{enumerate}

\item Title - 10 Words Max:
\begin{itemize}
\item A Study of Adaptive Finite Element Methods, and its Applications.
\item A Study of Functional Spaces, and Adaptive Finite Element Methods.
\end{itemize}

\item Overview/Abstract - 300 Words Max: \\
By definition, Partial Differential Equations (PDEs) are equations which impose relationships
between the partial derivatives of a multi-variable function. Adaptive Finite Element Methods (AFEMs)
are fundamental numerical instruments used to approximate PDEs, as these equations often do not have
analytic/exact solutions. 
With these definitions in mind, the aims of this project are to first conduct a study into the Functional Spaces, such as
Hilbert and Sobolev Spaces, which are relevant
to the analysis of PDEs. Then, we will progress into the study of Numerical Methods induced by weak formulations
of PDEs, namely, Finite Element Methods. Then finally, we will study state of the art Adaptive
Finite Element Methods by researching and applying relevant numerical techniques from a survey in this topic.

\item Originality, Creativity, and Significance - 300 Words Max: \\
Some text

\item Potential Benefit - 250 Words Max: \\
Some text

\item Relevant Experience - 250 Words Max: \\
Some text

\end{enumerate}

\end{document}
