\documentclass[11pt]{article}
%\setlength{\textwidth}{430pt}\setlength{\oddsidemargin}{11pt}
\usepackage{amssymb}
\usepackage{amsthm}
\usepackage{amsmath}
\usepackage{enumerate}
\usepackage{fancyhdr}
\usepackage{hyperref}
\hypersetup{colorlinks=true,linktoc=all,linkcolor=blue}
\usepackage[bottom=0.9in,top=0.9in]{geometry}

\theoremstyle{definition}
\newtheorem*{theorem}{Theorem}
\newtheorem*{definition}{Definition}
\newtheorem*{remark}{Remark}
\newtheorem*{motivation}{Motivation}

\begin{document}
\pagestyle{fancy}
\fancyhead[L]{}
\fancyhead[R]{}
\fancyhead[C]{\textbf{2023 Summer Research Notes}}
\tableofcontents
\newpage
\fancyhead[L]{\leftmark}
\fancyhead[R]{\rightmark}
\fancyhead[C]{}
\section{Prior Concepts to Know}
\subsection{Important inequalities}
\begin{theorem}[H\"{o}lder's inequality]
Assume $1 \leq p$, $q \leq \infty$, and $\frac{1}{p} + \frac{1}{p} = 1$. If $u \in L^p(\mathbb{R}^n)$ and $v \in L^q(\mathbb{R}^n)$, then
	\[\int_{\mathbb{R}^n}|uv|dx \leq \|u\|_{L^p(\mathbb{R}^n)}\|v\|_{L^q(\mathbb{R}^n)}.\]
\end{theorem}
\begin{proof}
	Letting $0 < t \in \mathbb{R}$, we can say
	\[|uv| = |ut||v/t|,\]
	and by Young's inequality
	\[|uv| = |ut||v/t| \leq \frac{|ut|^p}{p} + \frac{|v/t|^q}{q} = \frac{|u|^pt^p}{p} + \frac{|u|^q}{qt^q}.\]
	Now integrating with respect to $x \in \mathbb{R}^n$, results in
	\begin{equation}
		\int_{\mathbb{R}^n}|uv|dx \leq  \int_{\mathbb{R}^n}\left(\frac{|u|^pt^p}{p} + \frac{|u|^q}{qt^q}\right)dx
		= \frac{t^p}{p}\int_{\mathbb{R}^n}|u|^pdx + \frac{1}{qt^q}\int_{\mathbb{R}^n}|v|^qdx = g(t).
	\end{equation}
	For the sake of simplicity, let
	\[a = \int_{\mathbb{R}^n}|u|^pdx, \text{ and } b = \int_{\mathbb{R}^n}|v|^qdx.\]
	Now we wish to find the smallest $t > 0$ that satisfies (1). To do this we can employ Calculus, which says the minimum $t$
	we are looking for, denoted $t_0$, satisfies $g^{\prime}(t_0) = 0$. First, we need to compute $g^{\prime}(t)$ as follows
	\[g^{\prime}(t) = \frac{d}{dt}\left(\frac{t^p}{p}a + \frac{1}{qt^q}b\right) = t^{p-1}a - \frac{b}{t^{q+1}}.\]
	Then, let $g^{\prime}(t) = 0$ and solve for $t$ as follows
	\begin{equation*}
		\begin{aligned}
			&g^{\prime}(t) = t^{p-1}a - \frac{b}{t^{q+1}} = 0 \\
			&\Rightarrow t^{p-1}a = \frac{b}{t^{q+1}} \\
			&\Rightarrow t^{p-1}t^{q+1} = \frac{b}{a} \\
			&\Rightarrow t^{p+q} = \frac{b}{a} \\
			&\Rightarrow t = \left(\frac{b}{a}\right)^{\frac{1}{p+q}} = t_0.
		\end{aligned}
	\end{equation*}
	Finally, compute $g(t_0)$ as follows
	\begin{equation*}
		\begin{aligned}
			g(t_0) &= \frac{\left(\left(\frac{b}{a}\right)^{\frac{1}{p+q}}\right)^pa}{p} + \frac{b}{q\left(\left(\frac{b}{a}\right)^{\frac{1}{p+q}}\right)^q}b \\
			       &= \frac{\left(\frac{b}{a}\right)^{\frac{p}{p+q}}a}{p} + \frac{b}{q\left(\frac{b}{a}\right)^{\frac{q}{p+q}}} \\
			       &= \frac{\left(\frac{b}{a}\right)^{\frac{p-1}{p}}a}{p} + \frac{b}{q\left(\frac{b}{a}\right)^{\frac{1}{p}}} \\
			       &= \frac{b^{\frac{1}{q}}a^{\frac{1}{p}}}{p} + \frac{b^{\frac{1}{q}}a^{\frac{1}{p}}}{q} \\
			       &= b^{\frac{1}{q}}a^{\frac{1}{p}}\left(\frac{1}{p} + \frac{1}{q}\right) \\
			       &= b^{\frac{1}{q}}a^{\frac{1}{p}} \\
			       &= \left(\int_{\mathbb{R}^n}|v|^qdx\right)^{\frac{1}{q}}\left(\int_{\mathbb{R}^n}|u|^pdx\right)^{\frac{1}{p}} \\
			       &= \|u\|_{L^p(\mathbb{R}^n)}\|v\|_{L^q(\mathbb{R}^n)}.
		\end{aligned}
	\end{equation*}
	
\end{proof}
%\subsection{Linear Spaces}
%\subsection{Normed Linear Spaces}
%\subsection{Metric Spaces}
%\subsection{Banach Spaces}
%\subsection{$L^p$ Spaces}
%\subsection{Hilbert Spaces}
%\subsection{Weak Derivatives}
%\subsection{Mollifiers}
%\subsection{Partition of Unity}

\newpage

\section{H\"{o}lder Spaces}
\subsection{H\"{o}lder continuous functions}
\subsection{H\"{o}lder spaces are Banach spaces}

\newpage

\section{Sobolev Spaces}

\newpage

\subsection{Approximation}
\begin{motivation}
The main motivation of studying Sobolev spaces in general is that functions in these spaces can be approximated by smooth functions.
These approximations can be done locally, globally, and globally up to and including the boundary.
As expected these require three different Theorems (taken from Evans) which increase in difficulty to prove.
\end{motivation}

\begin{theorem}[Local approximation by smooth functions]
Let $U \subset \mathbb{R}^n$ be a bounded open set. Suppose $u \in W^{k,p}(U)$ for some $1 \leq p < \infty$, and set
	\[u^{\epsilon} = \eta_{\epsilon} * u \text{ in } U_{\epsilon}.\]
Then
	\begin{enumerate}[(i)]
		\item $u^{\epsilon} \in C^{\infty}(U_{\epsilon}) \,\,\, \forall \, \epsilon > 0$, and
		\item $u^{\epsilon} \rightarrow u \text{ in } W_{\text{loc}}^{k,p}(U)$, as $\epsilon \rightarrow 0$.
	\end{enumerate}
\end{theorem}

\begin{theorem}[Global approximation by smooth functions]
	Let $U \subset \mathbb{R}^n$ be a bounded open set. Suppose $u \in W^{k,p}(U)$ for some $1 \leq p < \infty$. Then there exists functions $u_m \in C^{\infty}(U)\cap W^{k,p}(U)$ such that
	\[u_m \rightarrow u \text{ in } W^{k,p}(U).\]
\end{theorem}
\begin{remark}
	Since $U \subset \mathbb{R}^n$ is a bounded open set, this Theorem says
	nothing about the smoothness of the boundary of $U$, which is the goal of the next Theorem.
\end{remark}

\begin{theorem}[Global approximation by functions smooth up to the boundary]
	Let $U \subset \mathbb{R}^n$ be a bounded open set. Suppose $\partial U$ is $C^1$ and suppose $u \in W^{k,p}(U)$ for some $1 \leq p < \infty$.
	Then there exists functions $u_m \in C^{\infty}(\overline{U})$ such that
	\[u_m \rightarrow u \text{ in } W^{k,p}(U).\]
\end{theorem}

\newpage

\subsection{Extensions}
\begin{motivation}
The motivation here is the desire to extend functions in $W^{k,p}(U)$ to functions in $W^{k,p}(\mathbb{R}^n)$.
Evans discusses and proves extending functions in $W^{1,p}(U)$ to functions in $W^{1,p}(\mathbb{R}^n)$ which becomes
useful for some of the Sobolev approximation proofs.
\end{motivation}

\begin{theorem}
	Suppose $1 \leq p \leq \infty$.
Let $U \subset \mathbb{R}^n$ be a bounded opend set, where $\partial U$ is $C^1$.
Let $V \subset \mathbb{R}^n$ be a bounded open set such that $U \subset\subset V$.
Then there exists a bounded linear operator
	\[E: W^{1,p}(U) \rightarrow W^{1,p}(\mathbb{R}^n)\]
such that $\forall u \in W^{1,p}(U)$:
	\begin{enumerate}[(i)]
		\item $Eu = u$ a.e. in $U$,
		\item $Eu$ has support within $V$, and
		\item \[\|Eu\|_{W^{1,p}(\mathbb{R}^n)} \leq C\|u\|_{W^{1,p}(U)},\]
		with $C$ depending only on $p$, $U$, and $V$.
	\end{enumerate}
\end{theorem}


\newpage

\subsection{Sobolev inequalities}
\begin{motivation}
The main goal of Sobolev inequalities is to find embeddings of Sobolev spaces in other spaces that may be easier to work with, more desirable,
or simply imply nice properties. These inequalities essentially provide estimates for functions in Sobolev spaces using other
well know function spaces. Many of the following proofs are taken from Evans.
\end{motivation}
\begin{theorem}[Gagliardo-Nirenberg-Sobolev Inequality]
Suppose $1\leq p < n$. There exists a constant $C > 0$, depending only on $p$ and $n$, such that
	\[\|u\|_{L^{p^*}(\mathbb{R}^n)} \leq C\|Du\|_{L^{p}(\mathbb{R}^n)},\]
for all $u \in C_{c}^{1}(\mathbb{R}^n)$.
\end{theorem}

\begin{theorem}[Estimates for $W^{1,p}$, where $1 \leq p < n$]
Let $U \subset \mathbb{R}^n$ be a bounded open set, and suppose $\partial U$ is $C^1$.
Suppose $u \in W^{1,p}(U)$ for some $1 \leq p < n$. It then follows that $u \in L^{p^*}$, with the estimate that
	\[\|u\|_{L^{p^*}(U)} \leq C\|u\|_{W^{1,p}(U)},\]
where $C > 0$ is constant and only depending on $p$, $n$, and $U$.
\end{theorem}

\begin{theorem}[Estimates for $W_{0}^{1,p}$, where $1 \leq p < n$]
Let $U \subset \mathbb{R}^n$ be a bounded open set.
	Suppose $u \in W_{0}^{1,p}(U)$ for some $1 \leq p < n$. Then we get the estimate
	\[\|u\|_{L^{q}(U)} \leq C\|Du\|_{L^{p}(U)},\]
for all $1 \leq q \leq p^*$, where $C > 0$ is constant and only depending on $p$, $q$, $n$, and $U$.
\end{theorem}
\begin{remark}
If $U$ is bounded as we supposed, then on $W_{0}^{1,p}$, the norm $\|Du\|_{L^{p}} \equiv \|u\|_{W^{1,p}(U)}$,
meaning this is still an estimate for functions in Sobolev spaces.
\end{remark}

\begin{theorem}[Morrey's Inequality]
Suppose $n < p \leq \infty$. Then there exists a constant $C > 0$, depending only on $p$ and $n$, such that
	\[\|u\|_{C^{0,\gamma}(\mathbb{R}^n)} \leq C\|u\|_{W^{1,p}(\mathbb{R}^n)},\]
for all $u \in W^{1,p}(U)$ with $u \in C^1(\mathbb{R}^n)$, where $\gamma = 1 - \frac{n}{p}$.
\end{theorem}

\begin{theorem}[Estimates for $W^{1,p}$, where $n < p \leq \infty$]
Let $U \subset \mathbb{R}^n$ be a bounded open set, and suppose $\partial U$ is $C^1$.
	Suppose $u \in W^{1,p}(U)$ for some $n < p \leq \infty$. It then follows that a continuous version $u^* \in C^{0,\gamma}(\overline{U})$,
where $\gamma = 1 - \frac{n}{p}$, with the estimate that
	\[\|u^*\|_{C^{0,\gamma}(\overline{U})} \leq C\|u\|_{W^{1,p}(U)},\]
where $C > 0$ is constant and only depending on $p$, $n$, and $U$.
\end{theorem}

\newpage

\section{Notation}
\begin{enumerate}[(i)]
	\item A \textit{multiindex} is a vector $\alpha = (\alpha_1, \dots, \alpha_n)$ where each component $\alpha_i \in \mathbb{N}_0$.
A multiindex has an order defined by
\[|\alpha| = \alpha_1 + \dots + \alpha_n.\]

\item Using our definition of a multiindex and letting $u(x)$ be some function, we define
\[D^{\alpha}u(x) = \frac{\partial^{|\alpha|}u(x)}{\partial x_1^{\alpha_1} \dots \partial x_n^{\alpha_n}} = \partial_{x_1}^{\alpha_1}\dots\partial_{x_n}^{\alpha_n}u(x).\]

\item Let $U,V \subset \mathbb{R}^n$. Then define
\[V \subset\subset U\]
to be when $V \subset \overline{V} \subset U$ and $\overline{V}$ is compact. In plain english this means $V$ is \textit{compactly contained} in $U$.

\item Let $f$ and $g$ be functions. Then define $*$ to be the \textit{convolution operator} where
\[(f*g)(x) = \int_{-\infty}^{\infty}{f(\tau)g(x - \tau)d\tau} = \int_{-\infty}^{\infty}{f(x - \tau)g(\tau)d\tau}\]
is the \textit{convolution} of the functions $f$ and $g$ which results in a third function that expresses
how one of the functions modifies the other. Note that I am assuming $f$ and $g$ are both supported on an
infinite interval, which may not always be the case.

\item Let $u(x)$ be a function where $x \in \mathbb{R}^n$. Then the \textit{gradient vector} of $u$ is 
	\[Du(x_1,\dots,x_n) = (u_{x_1},...,u_{x_n}).\]
\end{enumerate}


\end{document}
