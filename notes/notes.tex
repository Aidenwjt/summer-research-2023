\documentclass[11pt]{article}
%\setlength{\textwidth}{430pt}\setlength{\oddsidemargin}{11pt}
\usepackage{amssymb}
\usepackage{amsthm}
\usepackage{amsmath}
\usepackage{enumerate}
\usepackage{fancyhdr}
\usepackage{hyperref}
\hypersetup{colorlinks=true,linktoc=all,linkcolor=blue}
\usepackage[bottom=0.9in,top=0.9in]{geometry}

\theoremstyle{definition}
\newtheorem*{theorem}{Theorem}
\newtheorem*{definition}{Definition}
\newtheorem*{remark}{Remark}
\newtheorem*{motivation}{Motivation}

\begin{document}
\pagestyle{fancy}
\fancyhead[L]{}
\fancyhead[R]{}
\fancyhead[C]{\textbf{2023 Summer Research Notes}}
\tableofcontents
\newpage
\fancyhead[L]{\leftmark}
\fancyhead[R]{\rightmark}
\fancyhead[C]{}
\section{Prior Concepts to Know}
\begin{motivation}
The following are just some concepts, ideas, and interesting topics (mainly in analysis) that help in understanding these notes.
The main focus here is on background material required for studying Sobolev spaces and adaptive finite element methods.
\end{motivation}

\subsection{Important inequalities}
\begin{motivation}
A lot of analysis proofs in general can be boiled down to figuring out if some element is smaller than, bigger than, or equal to some other element.
For this reason, it is very helpful to know some of these important inequalities. 
\end{motivation}
\begin{theorem}[Young's inequality]
Let $p > 1$, $q < \infty$, and $\frac{1}{p}+\frac{1}{q} = 1$. Then
\[ab \leq \frac{a^p}{p} + \frac{b^q}{q}.\]
\end{theorem}
\begin{proof}
Suppose we $t = \frac{1}{q}$ and $(1-t)=\frac{1}{p}$, implying $t \in (0,1)$. Then since we know $\ln$ is a concave function, we have
	\begin{equation*}
		\begin{aligned}
			\ln{((1-t)a^p + tb^q)} &\geq (t-1)\ln{(a^p)} + t\ln{(b^q)} \\
					       &= p(t-1)\ln{(a)} + qt\ln{(b)} \\
					       &= \ln{(a)} + \ln{(b)} \\
					       &= \ln{(ab)}.
		\end{aligned}
	\end{equation*}
\end{proof}

\begin{theorem}[H\"{o}lder's inequality]
Assume $1 \leq p$, $q \leq \infty$, and $\frac{1}{p} + \frac{1}{p} = 1$. If $u \in L^p(\mathbb{R}^n)$ and $v \in L^q(\mathbb{R}^n)$, then
	\[\int_{\mathbb{R}^n}|uv|dx \leq \|u\|_{L^p(\mathbb{R}^n)}\|v\|_{L^q(\mathbb{R}^n)}.\]
\end{theorem}
\begin{proof}
	Letting $0 < t \in \mathbb{R}$, we can say
	\[|uv| = |ut||v/t|,\]
	and by Young's inequality
	\[|uv| = |ut||v/t| \leq \frac{|ut|^p}{p} + \frac{|v/t|^q}{q} = \frac{|u|^pt^p}{p} + \frac{|u|^q}{qt^q}.\]
	Now integrating with respect to $x \in \mathbb{R}^n$, results in
	\begin{equation}
	\label{eq:1}
		\int_{\mathbb{R}^n}|uv|dx \leq  \int_{\mathbb{R}^n}\left(\frac{|u|^pt^p}{p} + \frac{|u|^q}{qt^q}\right)dx
		= \frac{t^p}{p}\int_{\mathbb{R}^n}|u|^pdx + \frac{1}{qt^q}\int_{\mathbb{R}^n}|v|^qdx = g(t).
	\end{equation}
	For the sake of simplicity let
	\[a = \int_{\mathbb{R}^n}|u|^pdx, \text{ and } b = \int_{\mathbb{R}^n}|v|^qdx.\]
	Now we wish to find the smallest $t > 0$ that satisfies equation \ref{eq:1}. To do this we can employ Calculus, which says the minimum $t$
	we are looking for, which we denote as $t_0$, satisfies $g^{\prime}(t_0) = 0$. First, we need to compute $g^{\prime}(t)$ as follows
	\[g^{\prime}(t) = \frac{d}{dt}\left(\frac{t^p}{p}a + \frac{1}{qt^q}b\right) = t^{p-1}a - \frac{b}{t^{q+1}}.\]
	Then, let $g^{\prime}(t) = 0$ and solve for $t$ as follows
	\begin{equation*}
		\begin{aligned}
			&g^{\prime}(t) = t^{p-1}a - \frac{b}{t^{q+1}} = 0 \\
			&\Rightarrow t^{p-1}a = \frac{b}{t^{q+1}} \\
			&\Rightarrow t^{p-1}t^{q+1} = \frac{b}{a} \\
			&\Rightarrow t^{p+q} = \frac{b}{a} \\
			&\Rightarrow t = \left(\frac{b}{a}\right)^{\frac{1}{p+q}} = t_0.
		\end{aligned}
	\end{equation*}
	Finally, compute $g(t_0)$ as follows
	\begin{equation*}
		\begin{aligned}
			g(t_0) &= \frac{\left(\left(\frac{b}{a}\right)^{\frac{1}{p+q}}\right)^p}{p}a + \frac{1}{q\left(\left(\frac{b}{a}\right)^{\frac{1}{p+q}}\right)^q}b \\
			       &= \frac{\left(\frac{b}{a}\right)^{\frac{p}{p+q}}a}{p} + \frac{b}{q\left(\frac{b}{a}\right)^{\frac{q}{p+q}}} \\
			       &= \frac{\left(\frac{b}{a}\right)^{\frac{p-1}{p}}a}{p} + \frac{b}{q\left(\frac{b}{a}\right)^{\frac{1}{p}}} \\
			       &= \frac{b^{\frac{1}{q}}a^{\frac{1}{p}}}{p} + \frac{b^{\frac{1}{q}}a^{\frac{1}{p}}}{q} \\
			       &= b^{\frac{1}{q}}a^{\frac{1}{p}}\left(\frac{1}{p} + \frac{1}{q}\right) \\
			       &= b^{\frac{1}{q}}a^{\frac{1}{p}} \\
			       &= \left(\int_{\mathbb{R}^n}|v|^qdx\right)^{\frac{1}{q}}\left(\int_{\mathbb{R}^n}|u|^pdx\right)^{\frac{1}{p}} \\
			       &= \|v\|_{L^q(\mathbb{R}^n)}\|u\|_{L^p(\mathbb{R}^n)}.
		\end{aligned}
	\end{equation*}
\end{proof}

\newpage

\begin{theorem}[Cauchy-Schwarz inequality]
	Let $u$ and $v$ be arbitrary vectors in an inner product space over the scalar field $\mathbb{F}$,
	where $\mathbb{F}$ is either $\mathbb{R}$ or $\mathbb{C}$. Then
	\begin{equation*}
		|(u,v)| \leq \|u\|\|v\|
	\end{equation*}
	with equality holding (``='') if and only if $u$ and $v$ are linearly dependent.
	Moreover, if $|(u,v)| = \|u\|\|v\|$ and $v\not=0$, then
	\begin{equation*}
		u = \frac{(u,v)}{\|v\|^2}v.
	\end{equation*}
\end{theorem}
\begin{proof}
	When $u=0$ or $v=0$ then the inequality is trivial, so assume WLOG that $v\not=0$.
	Let
	\begin{equation*}
		z = u - \frac{(u,v)}{(v,v)}v,
	\end{equation*}
	then it follows from the linearity of the inner product that
	\begin{equation*}
		(z,v) = \left(u - \frac{(u,v)}{(v,v)}v\right) = (u,v) - \left(\frac{(u,v)}{(v,v)}v,v\right) = (u,v) - \frac{(u,v)}{(v,v)}(v,v) = 0.
	\end{equation*}
	This implies $z$ is orthogonal to $v$. Now applying Pythagorean's Theorem to
	\begin{equation*}
		u = \frac{(u,v)}{(v,v)}v + z
	\end{equation*}
	results in
	\begin{equation*}
		\begin{aligned}
			\|u\|^2 &= \left\|\frac{(u,v)}{(v,v)}v\right\|^2 + \|z\|^2 \\
				&= \left|\frac{(u,v)}{(v,v)}\right|^2 \|v\|^2 + \|z\|^2 \\
				&= \frac{|(u,v)|^2}{|(v,v)^{1/2}(v,v)^{1/2}|^2}\|v\|^2 + \|z\|^2 \\
				&= \frac{|(u,v)|^2}{|\|v\|^2|^2}\|v\|^2 + \|z\|^2 \\
				&= \frac{|(u,v)|^2}{\|v\|^2} + \|z\|^2 \\
				&\geq \frac{|(u,v)|^2}{\|v\|^2}.
		\end{aligned}
	\end{equation*}
	Multiplying through by $\|v\|^2$ results in our desired inequality. We can also observe if $z=0$, we get
	\begin{equation*}
		u = \frac{(u,v)}{(v,v)}v = \frac{(u,v)}{(v,v)^{1/2}(v,v)^{1/2}}v = \frac{(u,v)}{\|v\|^2}v,
	\end{equation*}
	as desired.
\end{proof}

\newpage

\subsection{Banach Spaces}
\begin{motivation}
Banach spaces are very important in the study of Sobolev spaces as Sobelev spaces themselves are Banach spaces, which will be proven later on.
Being able to prove a space is a Banach space is very helpful because Banach spaces are very nice mathematical structures that imply
a lot of nice properties, as we will see.
\end{motivation}
\begin{definition}
	A \textit{linear space} over a field $F$ (usually a subfield of the complex numbers) is a non-empty set $X$ equipped with a special element 0, 
	three operations, and a set of
	axioms that the three operations must satisfy. These operations and axioms are laid out nicely for the real numbers by Professor Tao
	\href{https://www.math.ucla.edu/~tao/resource/general/121.1.00s/vector_axioms.html}{here}.
\end{definition}
\begin{definition}
	A \textit{normed linear space} is a linear space $X$ over a subfield $F$ of the complex numbers equipped with a norm,
	where the norm is a real-valued function $\|\cdot\| : X \rightarrow \mathbb{R}$ satisfying the following four axioms:
	\begin{enumerate}[(i)]
		\item $\|u+v\| \leq \|u\|+\|v\|$ for all $u,v \in X$.
		\item $\|\lambda u\| = |\lambda|\|u\|$ for all $u \in X$ and $\lambda \in F$.
		\item $\|u\| \geq 0$ for all $u \in V$.
		\item $\|u\| = 0$ if and only if $u=0$.
	\end{enumerate}
	A normed linear space $X$ is also a \textit{metric space} under the metric $\rho$ defined by
	\[\rho(u,v) = \|u-v\|\]
	for all $u,v \in X$.
\end{definition}
\begin{definition}
	A \textit{complete metric space} is a metric space $X$ where every Cauchy sequence of points in $X$ converge to a point also in $X$.
	i.e. a space that has no missing points.
\end{definition}
\begin{definition}
	A \textit{Banach space} is a complete normed linear space.
\end{definition}

\newpage

\subsection{Hilbert spaces}
\begin{motivation}
	Hilbert spaces are important in the study of Sobolev spaces as some Sobolev spaces are themselves Hilbert spaces.
	Hilbert spaces are also important as there are some very important representation theorems for boundary value problems that involve Hilbert spaces.
\end{motivation}
Let $H$ be a real linear space.
\begin{definition}
	Let $H$ be a real linear space. A mapping $(\cdot , \cdot) : H \times H \rightarrow \mathbb{R}$ is called an \textit{inner product} if
	\begin{enumerate}
		\item $(u,v) = (v,u)$ for all $u,v \in H$,
		\item $u \mapsto (u,v)$ is linear for each $v \in H$,
		\item $(u,v) \geq 0$ for all $v \in H$,
		\item $(u,u) = 0$ if and only if $u = 0$.
	\end{enumerate}
\end{definition}
\begin{definition}
	The inner product is also related to the norm of $H$ as
	\begin{equation}
		\label{eq:2}
		\|u\| = (u,u)^{1/2},
	\end{equation}
	and by the Cauchy-Schwarz inequality, we have
	\begin{equation}
		\label{eq:3}
		|(u,v)| \leq \|u\|\|v\|
	\end{equation}
	for all $u,v \in H$. Equation \ref{eq:3} verifies that equation \ref{eq:2} defines a norm on $H$. 
\end{definition}
\begin{definition}
	A \textit{Hilbert space} $H$ is a Banach space equipped with an inner product which defines the norm on $H$.
\end{definition}

\begin{theorem}[Riesz Representation Theorem]
	For every bounded linear functional $F$ on a Hilbert space $H$, there exists a uniquely determined element $f \in H$
	such that $F(x) = (x,f)$ for all $x \in H$. Furthermore, $\|F\|_{\text{op}} = \|f\|_H$.
\end{theorem}
\begin{proof}
	For the case when $H = N(F) = \{x \in H \, | \, F(x) = 0\}$ we find that our uniquely determined element is $f=0$ as the zero vector is
	orthogonal to everything including itself, implying $F(x) = 0 = (x,0)$ for all $x \in H$. However, assume $H \not= N(F)$.
	$N(F)$ is now a closed subspace of $H$, which by virtue of the Projection Theorem implies there exists non-zero elements in $N(F)^{\bot}$.
	Take $z \in N(F)^{\bot}$ such that $z \not= 0$, where it follows that $(u,z) = 0$ for all $u \in N(F)$. Furthermore, $F(z) \not= 0$ and for any $x \in H$
	\[F\left(x - \frac{F(x)}{F(z)}z\right) = F(x) - \frac{F(x)}{F(z)}F(z) = 0,\]
	implying that $x - \frac{F(x)}{F(z)}z \in N(F)$, which further implies that $\left(x - \frac{F(x)}{F(z)}z, z\right) = 0$.
	Now by the properties of inner products, we have
	\begin{equation*}
		\begin{aligned}
			(x,z) &= \left(x - \frac{F(x)}{F(z)}z,z\right) + \left(\frac{F(x)}{F(z)}z,z\right) \\
			      &= \left(\frac{F(x)}{F(z)}z,z\right) \\
			      &= \frac{F(x)}{F(z)}(z,z) \\
			      &= \frac{F(x)}{F(z)}(z,z)^{1/2}(z,z)^{1/2} \\
			      &= \frac{F(x)}{F(z)}\|z\|^2
		\end{aligned}
	\end{equation*}
	for any $x \in H$. Rearranging the above equation results in 
	\[F(x) = (x,z)\frac{F(z)}{\|z\|^2} = \left(x,\frac{F(z)}{\|z\|^2}z\right),\]
	where we now just need to show $f = \frac{F(z)}{\|z\|^2}z$ is unique.
	To prove the uniqueness of $f$ assume there exists $f_1,f_2 \in H$ such that $F(x) = (x,f_1) = (x,f_2)$ for all $x \in H$,
	and assume we have $x \in H$ such that $\|x\| > 0$. We can see that this implies
	\begin{equation*}
		0 = |F(x) - F(x)| = |(x,f_1) - (x,f_2)| = |(x,f_1-f_2)| \leq \|x\|\|f_1 - f_2\|.
	\end{equation*}
	This implies $\|f_1-f_2\| = 0$, which futher implies $f_1-f_2= 0$, proving the uniqueness of $f$.
	Finally, to show $\|F\|_{\text{op}} = \|f\|_H$ we can use the fact that
	\begin{equation*}
		\|F\|_{\text{op}} = \sup\left\{\frac{\|F(x)\|_{\mathbb{R}}}{\|x\|_H}\, : \, x \not= 0 \text{ and } x \in H\right\}
	\end{equation*}
	to say
	\begin{equation*}
		\|F\|_{\text{op}} = \sup_{x\not=0}{\frac{\|F(x)\|_{\mathbb{R}}}{\|x\|_H}} = \sup_{x\not=0}{\frac{|F(x)|}{\|x\|_H}} = \sup_{x\not=0}{\frac{|(x,f)|}{\|x\|_H}} \leq \sup_{x\not=0}{\frac{\|x\|_H\|f\|_H}{\|x\|_H}} = \|f\|_H.
	\end{equation*}
	We can also say
	\begin{equation*}
		\|f\|_H^2 = \|f\|_H\|f\|_H = (f,f)^{1/2}(f,f)^{1/2} = (f,f) = F(f) \leq \|F(f)\|_{\text{op}}\|f\|_H.
	\end{equation*}
	These inequalities imply that $\|F\|_{\text{op}} \leq \|f\|_H$ and $\|f\|_H \leq \|F\|_{\text{op}}$, respectively, which can only be true if $\|F\|_{\text{op}} = \|f\|_H$.
\end{proof}

\begin{theorem}[Lax-Milgram Theorem]
	Let $B$ be a bounded, coercive bilinear form on a Hilbert space $H$. Then for every bounded linear functional $F$ on $H$
	there exists a uniquely determined element $f\in H$ such that
	\begin{equation*}
		B(x,f) = F(x)
	\end{equation*}
	for all $x \in H$.
\end{theorem}
\begin{proof}
	For any $f \in H$, define the mapping 
	\begin{equation}
		\label{eq:5}
		f \mapsto (F = B(\cdot, f)).
	\end{equation}
	Then since $F = B(\cdot, f) \in H^*$, we can use the Riesz Representation Theorem (RRT) to  define another mapping
	\begin{equation}
		\label{eq:6}
		(F = B(\cdot,f)) \stackrel{RRT}{\mapsto} \tilde{f},
	\end{equation}
	where $\tilde{f}$ is defined to be a uniquely determined element in $H$.
	Combining mappings \ref{eq:5} and \ref{eq:6} allows us to define one more mapping
	\begin{equation}
		\label{eq:7}
		f \mapsto \tilde{f},
	\end{equation}
	where we denote this mapping as $T$ such that $Tf = \tilde{f}$.
	With these mapping we now have
	\begin{equation*}
		F(z) = B(z,f) = (z,Tf)
	\end{equation*}
	for any $z \in H$.
	This may look like we are done, but notice that we have no idea if $f$ is unique or not.
	So our goal now is to show that $T$ is a bijective bounded linear operator from $H$ to $H$,
	which will allow us to use the Bounded Inverse Theorem to say $T^{-1}$ exists and is also bounded.
	Firstly, by the linearity and bilinearity of the inner product and the bilinear form $B$, respectively,
	we can say that for all $x,y,z \in H$ and $\lambda, \mu \in \mathbb{C}$, we have
	\begin{equation*}
		\begin{aligned}
			(z,T(\lambda x + \mu y)) &= B(z, \lambda x + \mu y) \\
						 &= B(z,\lambda x) + B(z, \mu y) \\
						 &= \lambda B(z,x) + \mu B(z,y) \\
						 &= \lambda (z,Tx) + \mu (z,Ty) \\
						 &= (z,\lambda Tx) + (z, \mu Ty) \\
						 &= (z, \lambda Tx + \mu Ty),
		\end{aligned}
	\end{equation*}
	implying that $T(\lambda x + \mu y) = \lambda Tx + \mu Ty$, which is the definition of a linear operator.
	Next, for any $z \in H$, we have
	\begin{equation*}
		|(z,Tf)| = |B(z,f)| \leq k\|z\|\|f\|
	\end{equation*}
	for some $k>0$, where letting $z = Tf$ results in
	\begin{equation*}
		\|Tf\|^2 = |(Tf,Tf)| = |B(Tf,f)| \leq k\|Tf\|\|f\|,
	\end{equation*}
	implying that
	\begin{equation}
		\label{eq:8}
		\|Tf\| \leq k\|f\|.
	\end{equation}
	Thus, $T$ is bounded. Now, also note that since
	\begin{equation*}
		(z,Tf) = B(z,f)
	\end{equation*}
	it follows that we can use the Cauchy-Schwarz inequality and the coercivity of $B$ to say that if $z=f$, then
	\begin{equation}
		\label{eq:9}
		\|f\|\|Tf\| \geq |(f,Tf)| = |B(f,f)| \geq |\nu \|f\|^2| = \nu \|f\|^2
	\end{equation}
	for some $\nu > 0$. Inequalities \ref{eq:8} and \ref{eq:9} imply
	\begin{equation}
		\label{eq:10}
		\nu \|f\| \leq \|Tf\| \leq k \|f\|
	\end{equation}
	for some $\nu >0$ and $k >0$.
	By definition $T$ is injective provided that for any $x,y \in H$, if $Tx = Ty$, then $x=y$.
	But note that the linearity of $T$ implies that we can instead say ``$T$ is injective provided that
	for all $w \in H$, if $Tw = 0$, the $w = 0$ .'' It is then easy to see that inequality \ref{eq:10}
	satisfies this definition, and thus $T$ is injective.
	Now to prove that $T$ is also surjective, we first need to prove that $T$ has a closed range so that
	we can invoke the Hilbert Projection Theorem. To prove this, let $T(H)$ denote the range of $T$.
	Then since $T(H)$ is a subspace of a Hilbert space we know that there exists a Cauchy sequence $\{\tilde{f}_k\}_{k=1}^{\infty} \subset H$
	such that $\tilde{f}_k \rightarrow \tilde{f}$ in $T(H)$, as $k\rightarrow \infty$. Then, since $T$ is injective, there exists a unique
	sequence $\{f_k\}_{k=1}^{\infty} \subset H$ such that $\tilde{f}_k = Tf_k$, for all $k \in \mathbb{N}$. Now $\{\tilde{f}_k\}_{k=1}^{\infty}$
	being Cauchy implies that for any $\epsilon > 0$ there exists $N >0$ such that
	\begin{equation*}
		\epsilon > \|\tilde{f}_k - \tilde{f}_{\ell}\| = \|Tf_k - Tf_{\ell}\| = \|T(f_k - f_{\ell})\| \geq \nu \|f_k - f_{\ell}\|
	\end{equation*}
	for any $k,\ell \in \mathbb{N}$ and some $\nu >0$.
	This implies $\{f_k\}_{k=1}^{\infty}$ is also Cauchy, further implying there exists a unqiue $f \in H$ such that
	$f_k \rightarrow f$ in $H$, as $k \rightarrow \infty$.
	Finally, to show $Tf = \tilde{f}$, observe that
	\begin{equation*}
		Tf = \lim_{k\rightarrow \infty}{Tf_k} = \lim_{k\rightarrow \infty}{\tilde{f}_k} = \tilde{f}.
	\end{equation*}
	Thus, $T$ has a closed range. Now to prove $T$ is surjective, suppose by contradiction that $T$ is not surjective.
	Then this implies $T(H)$ is a closed proper subspace of $H$, where we know by the Hilbert Projection Theorem that
	there exists non-zero elements $z \in T(H)^{\bot}$ such that $(z,Tf) = 0$. Letting $f = z$ results in
	\begin{equation*}
		0 = |(z,Tz)| = |B(z,z)| \geq \nu \|z\|^2 \geq 0
	\end{equation*}
	which can only be true if $z = 0$. Thus, $T$ must be surjective.
	Collecting ourselves, we have now proven that $T$ is a bijective bounded linear operator with a closed range.
	This means by virtue of the Bounded Inverse Theorem, $T^{-1}$ exists and is bounded.
	Then invoking the RRT again, we can say for any $F \in H^*$, there exists a uniquely determined element $\hat{f} \in H$ such that
	\begin{equation*}
		F(z) = (z,\hat{f})
	\end{equation*}
	for any $z \in H$. Now since we know
	\begin{equation*}
		F = B(\cdot,f)
	\end{equation*}
	for any $f \in H$, we can say for any $z \in H$
	\begin{equation*}
		F(z) = (z, TT^{-1}\hat{f}) = B(z,T^{-1}\hat{f})
	\end{equation*}
	because we have proven that $T$ is a bijection. Letting $T^{-1}\hat{f} = g$ results in
	\begin{equation*}
		F(z) = B(z,g)
	\end{equation*}
	for any $z \in H$. Now all that is left is to prove the uniqueness of $g$.
	Suppose we have $g_1,g_2 \in H$ such that $Tg_1$ and $Tg_2$ are unique elements satisfying
	the RRT. This means
	\begin{equation*}
		\begin{aligned}
			&B(z,g_1) = F(z) = B(z,g_2) \\
			&\Rightarrow B(z,g_1) = B(z,g_2) \\
			&\Rightarrow B(z,g_1) - B(z,g_2) = 0 \\
			&\Rightarrow B(z,g_1 - g_2) = 0.
		\end{aligned}
	\end{equation*}
	Letting $z= g_1 - g_2$ results in
	\begin{equation*}
		0 = B(g_1 - g_2, g_1 - g_2) \geq \nu \|g_1 - g_2\|^2 \geq 0
	\end{equation*}
	for some $\nu >0$. This can only be true if $g_1 - g_2 = 0$, proving the uniqueness of $g$.
\end{proof}

\newpage

\subsection{H\"{o}lder spaces}
\begin{motivation}
	H\"{o}lder spaces are function spaces which contain well-behaved functions, and where the function space itself is a Banach space.
	Functions being well-behaved in this case just means they are differentiable and continuous to certain degrees.
	H\"{o}lder spaces become relevant when discussing Sobolev inequalities.
\end{motivation}
Let $U \subset \mathbb{R}^n$ be a bounded open set and $0 < \gamma \leq 1$.
\begin{definition}[H\"{o}lder continuous functions]
\textit{Lipschitz continuous functions} $u : U \rightarrow \mathbb{R}$
satisfy the estimate
\[|u(x) - u(y)| \leq C|x-y|\]
where $x,y \in U$ and $C$ is some constant.
Now with the same assumptions, consider the variant
\[|u(x) - u(y)| \leq C|x-y|^{\gamma},\]
where $u$ is now said to be \textit{H\"{o}lder continuous with exponent $\gamma$}.
\end{definition}

\begin{definition}[Supremum norm and $\gamma^{th}$-H\"{o}lder seminorm]~ 
\begin{enumerate}[(i)]
\item If $u : U \rightarrow \mathbb{R}$ is bounded and continuous, the supremum norm is denoted
\[\|u\|_{C(\overline{U})} = \sup_{x \in U}{|u(x)|}.\]

\item The $\gamma^{th}$-H\"{o}lder seminorm of $u : U \rightarrow \mathbb{R}$ is
\[[u]_{C^{0,\gamma}(\overline{U})} = \sup_{\substack{x,y\in U \\ x\not= y}}\left\{\frac{|u(x) - u(y)|}{|x-y|^{\gamma}}\right\},\]
and the $\gamma^{th}$-H\"{o}lder norm is
\[\|u\|_{C^{0,\gamma}(\overline{U})} = \|u\|_{C(\overline{U})} + [u]_{C^{0,\gamma}(\overline{U})}.\]
\end{enumerate}
\end{definition}

\begin{definition}
The H\"{o}lder space $C^{k,\gamma}(\overline{U})$ consists of all functions $u \in C^k(\overline{U})$,
where the norm
\[\|u\|_{C^{k,\gamma}(\overline{U})} = \sum_{|\alpha| \leq k}\|D^{\alpha}u\|_{C(\overline{U})} + \sum_{|\alpha| = k}[D^{\alpha}u]_{C^{0,\gamma}(\overline{U})} < \infty.\]
\end{definition}

\begin{theorem}[H\"{o}lder spaces as function spaces]
The space of functions $C^{k,\gamma}(\overline{U})$ is a Banach space.
\end{theorem}
\begin{proof}
\href{https://github.com/Aidenwjt/some-math-notes/blob/master/exercises/evans/exercises.pdf}{Problem 5.1.}
\end{proof}

\newpage

\subsection{Weak Derivatives}
\begin{motivation}
	The motivation for weak derivatives in our case comes from the desire to satisfy certain integral relations given by the integration by parts formula.
	Let $U \subset \mathbb{R}^n$ be a bounded open set and let $u\in C^1(U)$. Then if $\phi \in C_{c}^{\infty}(U)$, we can see that
	\begin{equation*}
		\int_U u\phi_{x_i}dx = [u\phi]_U - \int_U \phi u_{x_i}dx = - \int_U \phi u_{x_i}dx
	\end{equation*}
	when we use the integration by parts formula. Now letting $k\in \mathbb{N}$, $u \in C^k(U)$, and $\alpha$ be a multi-index, we find that
	\begin{equation*}
		\begin{aligned}
			\int_U u D^{\alpha}\phi dx &= \int_U u \frac{\partial^{\alpha_1}}{\partial x_1^{\alpha_1}}\dots\frac{\partial^{\alpha_n}}{\partial x_n^{\alpha_n}}\phi dx \\
						   &= - \int_U \frac{\partial^{\alpha_n}}{\partial x_n^{\alpha_n}}u \frac{\partial^{\alpha_1}}{\partial x_1^{\alpha_1}}\dots\frac{\partial^{\alpha_{n-1}}}{\partial x_{n-1}^{\alpha_{n-1}}}\phi dx \\
						   &= \quad\qquad\qquad\vdots \\ 
						   &= (-1)^{|\alpha|}\int_U \phi\frac{\partial^{\alpha_1}}{\partial x_1^{\alpha_1}}\dots\frac{\partial^{\alpha_n}}{\partial x_n^{\alpha_n}}u dx \\
						   &= (-1)^{|\alpha|}\int_U \phi D^{\alpha}u dx.
		\end{aligned}
	\end{equation*}
	This gives us the relation
	\begin{equation}
		\label{eq:4}
		\int_U u D^{\alpha}\phi dx = (-1)^{|\alpha|}\int_U D^{\alpha}u\phi  dx
	\end{equation}
	where the LHS of equation \ref{eq:4} also holds if $u\in L_{\text{loc}}^1(U)$.
	The question now is ``does $u$ need to be k-times partially differentiable, or can we replace $D^{\alpha}u$ with something that also satisfies
	equation \ref{eq:4}, but is just locally summable instead?''.
\end{motivation}
\begin{definition}
	Let $u,v \in L_{\text{loc}}^1(U)$ and let $\alpha$ be a multi-index. Then we say that $v$ is the 
	\textit{$\alpha^{th}$ weak partial derivative of $u$} provided
	\begin{equation*}
		\int_U u D^{\alpha}\phi dx = (-1)^{|\alpha|}\int_U v \phi dx.
	\end{equation*}
	In this case, we also say $D^{\alpha}u = v$ ``in the weak sense.''
\end{definition}
\begin{definition}
	The space of k-times weakly differentiable functions is denoted $W^k(U)$, where clearly $C^k(U) \subset W^k(U)$.
\end{definition}

\newpage

\subsection{Mollifiers}
\begin{motivation}
	Mollifiers are smooth functions which allow us to ``smooth over'' other functions via convolution.
	Since mollifiers are smooth functions, we find that they become very important in the study of Sobolev spaces,
	especially for approximations, Sobolev inequalities, and the Sobolev embedding theorem.
\end{motivation}
Let $U \subset \mathbb{R}^n$ be a bounded open set.
\begin{definition}[Standard mollifier]~ 
	\begin{enumerate}[(i)]
		\item Define $\eta \in C^{\infty}(\mathbb{R}^n)$ by
			\begin{equation*}
				\eta(x) =
				\begin{cases}
					C e^{\frac{1}{|x|^2 - 1}},& \text{if } |x| < 1 \\
					0,& \text{if } |x| \geq 1,
				\end{cases}
			\end{equation*}
			where $C$ is chosen to satisfy
			\begin{equation*}
				\int_{\mathbb{R}^n}{\eta(x)dx} = 1.
			\end{equation*}
			$\eta$ is then called the \textit{standard mollifier}.
		\item Now for each $\epsilon > 0$, define
			\begin{equation*}
				\eta_{\epsilon}(x) = \frac{1}{\epsilon^n}\eta\left(\frac{x}{\epsilon}\right).
			\end{equation*}
			The functions $\eta_{\epsilon} \in C^{\infty}(\mathbb{R}^n)$ also satisfy
			\begin{equation*}
				\int_{\mathbb{R}^n}{\eta_{\epsilon}(x)dx} = 1,
			\end{equation*}
			where $support(\eta_{\epsilon}) \in B(0,\epsilon)$.
	\end{enumerate}
\end{definition}

\begin{definition}
	If $f : U \rightarrow \mathbb{R}$ is locally integrable, then define its \textit{mollification} as
	the convolution of $f$ and $\eta_{\epsilon}$ in $U_{\epsilon} = \{x\in U \,|\, dist(x,\partial U) > \epsilon \}$, written as
	\begin{equation*}
		f^{\epsilon} = f * \eta_{\epsilon} \text{ in } U_{\epsilon}.
	\end{equation*}
	In terms of the convolution, we can write this as
	\begin{equation*}
		f^{\epsilon}(x) = \int_U \eta_{\epsilon}(x-y)f(y)dy = \int_{B(0,\epsilon)} \eta_{\epsilon}(y)f(x-y)dy
	\end{equation*}
	for $x \in U_{\epsilon}$.
\end{definition}
\begin{theorem}[Properties of mollifiers]~ 
	\begin{enumerate}[(i)]
		\item $f^{\epsilon} \in C^{\infty}(U_{\epsilon})$.
		\item $f^{\epsilon} \rightarrow f$ a.e., as $\epsilon \rightarrow 0$.
		\item If $f \in C(U)$, then $f^{\epsilon} \rightarrow f$ uniformly on compact subsets of $U$.
		\item If $1\leq p < \infty$ and $f \in L_{\text{loc}}^p(U)$, then $f^{\epsilon} \rightarrow f$ in $L_{\text{loc}}^p(U)$.
	\end{enumerate}
\end{theorem}
\begin{proof}
	In progress...
\end{proof}

%\subsection{$L^p$ Spaces}
%\subsection{Hilbert Spaces}
%\subsection{Partition of Unity}

\newpage

\section{Sobolev Spaces}
\begin{motivation}
Sobolev spaces are motivated by the study of elliptic PDEs, for example, the Poisson equation
\[\Delta{u} = f.\]
Multiplying this equation on both sides by $\phi \in C_{c}^{\infty}$, then integrating both sides, then integrating the left side by parts, results in
\begin{equation*}
	\begin{aligned}
		\int \phi \Delta u &= \int \phi f \\
		\Rightarrow \int \phi \nabla^2 u &= \int \phi f \\
		\Rightarrow \phi \nabla u - \int \phi \nabla u &= \int \phi f \\ 
		\Rightarrow - \int \phi \nabla u &= \int \phi f.
	\end{aligned}
\end{equation*}
It can then be shown that for any $u \in C^2$, if the above integral relation holds, then $u$ is a classical solution of the original equation.
However, one may notice that the integral relation only needs $\nabla u \in L^p$, for some fixed $p$, to be satisfied.
Also, one may notice the similarity of the integral relation to the definition of weak derivatives. Both these observations lead to a more generalized
case where $u \in L^1_{\text{loc}}$ and $u \not\in C^1$, only requiring that the $\alpha^{th}$ weak partial derivative of $u$ belong to $L^p$, for some fixed $p$.
A space of functions like this is very useful as the functions in these spaces don't necessarily have to be continuous or totally differentiable,
yet they still satisfy the above relation. 
\end{motivation}

\subsection{Definition of the $W^{k,p}$ spaces}
\begin{motivation}
	We now desire to explicitly state the generalized formal definition for the space of functions described above,
	as well as layout some useful properties that come along with this space.
\end{motivation}
Fix $1 \leq p \leq \infty$, let $k \in \mathbb{N}$, and let $U \subset \mathbb{R}^n$ be a bounded open set.
\begin{definition}
	The Sobolev space
	\[W^{k,p}(U) = \{u \in L^1_{\text{loc}}(U) \, | \, D^{\alpha}u \text{ exists in the weak sense, and } D^{\alpha}u \in L^p(U) \, \forall \, |\alpha| \leq k\},\]
	consists of all locally summable functions $u : U \rightarrow \mathbb{R}$ such that for each $|\alpha| \leq k$, $D^{\alpha}u$ exists in the weak sense
	and belongs to $L^p(U)$.
\end{definition}
\begin{remark}
	When $p=2$ we will instead use the notation
	\[W^{k,2}(U) = H^k(U),\]
	because this Sobolev space is also a Hilbert space.
\end{remark}
\begin{definition}
	If $u \in W^{k,p}(U)$, we define its norm to be
	\[\|u\|_{W^{k,p}(U)} = \sum_{|\alpha| \leq k} \|D^{\alpha}u\|_{L^p(U)}.\]
\end{definition}
\begin{theorem}[Sobolev spaces as function spaces]
	For each $k \in \mathbb{N}$ and $1 \leq p \leq \infty$, the Sobolev space $W^{k,p}(U)$ is a Banach space.
\end{theorem}
\begin{proof}
	In progress...
\end{proof}

\newpage

\subsection{Approximation}
\begin{motivation}
Sobolev spaces are nice function spaces as we find that elements in Sobolev spaces can be approximated by sequences of smooth functions.
These approximations can be done locally, globally up to but excluding the boundary, and globally up to and including the boundary.
As expected these require three different Theorems (taken from Evans) which increase in difficulty to prove.
\end{motivation}
Let $U \subset \mathbb{R}^n$ be a bounded open set and suppose $u \in W^{k,p}(U)$ for some $1 \leq p < \infty$.
\begin{theorem}[Local approximation by smooth functions]
Set
	\[u^{\epsilon} = \eta_{\epsilon} * u \text{ in } U_{\epsilon}.\]
Then
	\begin{enumerate}[(i)]
		\item $u^{\epsilon} \in C^{\infty}(U_{\epsilon}) \,\,\, \forall \, \epsilon > 0$, and
		\item $u^{\epsilon} \rightarrow u \text{ in } W_{\text{loc}}^{k,p}(U)$, as $\epsilon \rightarrow 0$.
	\end{enumerate}
\end{theorem}
\begin{proof}
In progress...
\end{proof}

\begin{theorem}[Global approximation by smooth functions]
	There exists functions $u_m \in C^{\infty}(U)\cap W^{k,p}(U)$ such that
	\[u_m \rightarrow u \text{ in } W^{k,p}(U).\]
\end{theorem}
\begin{proof}
In progress...
\end{proof}

\begin{remark}
Since $U \subset \mathbb{R}^n$ is a bounded open set, this Theorem says
nothing about the smoothness of the boundary of $U$, which is the goal of the next Theorem.
\end{remark}

\begin{theorem}[Global approximation by functions smooth up to the boundary]
	Suppose $\partial U$ is $C^1$.
	Then there exists functions $u_m \in C^{\infty}(\overline{U})$ such that
	\[u_m \rightarrow u \text{ in } W^{k,p}(U).\]
\end{theorem}
\begin{proof}
In progress...
\end{proof}

\newpage

\subsection{Extensions}
\begin{motivation}
Naturally, we should desire to extend functions in $W^{k,p}(U)$ to functions in $W^{k,p}(\mathbb{R}^n)$.
Evans discusses and proves this extension which becomes useful for some of the Sobolev inequalities proofs.
\end{motivation}

\begin{theorem}
	Suppose $1 \leq p \leq \infty$.
Let $U \subset \mathbb{R}^n$ be a bounded open set, where $\partial U$ is $C^1$.
Let $V \subset \mathbb{R}^n$ be a bounded open set such that $U \subset\subset V$.
Then there exists a bounded linear operator
	\[E: W^{1,p}(U) \rightarrow W^{1,p}(\mathbb{R}^n)\]
such that $\forall u \in W^{1,p}(U)$:
	\begin{enumerate}[(i)]
		\item $Eu = u$ a.e. in $U$,
		\item $Eu$ has support within $V$, and
		\item $\|Eu\|_{W^{1,p}(\mathbb{R}^n)} \leq C(p,U,V)\|u\|_{W^{1,p}(U)}.$
	\end{enumerate}
\end{theorem}
\begin{proof}
In progress...
\end{proof}

\newpage

\subsection{Sobolev inequalities}
\begin{motivation}
Sobolev inequalities are useful for finding embeddings of Sobolev spaces in other function spaces that may be easier to work with, more desirable,
or simply imply nice properties. Many of the following theorems are taken from Evans.
\end{motivation}
Let $U \subset \mathbb{R}^n$ be a bounded open set.
\begin{theorem}[Gagliardo-Nirenberg-Sobolev inequality]
Suppose $1\leq p < n$. There exists a constant $C(p,n) > 0$ such that
	\[\|u\|_{L^{p^*}(\mathbb{R}^n)} \leq C(p,n)\|Du\|_{L^{p}(\mathbb{R}^n)},\]
for all $u \in C_{c}^{1}(\mathbb{R}^n)$.
\end{theorem}
\begin{proof}
In progress...
\end{proof}

\begin{theorem}[Estimates for $W^{1,p}$, where $1 \leq p < n$]
Suppose $\partial U$ is $C^1$,
and suppose $u \in W^{1,p}(U)$ for some $1 \leq p < n$. It then follows that $u \in L^{p^*}$, with the estimate that
	\[\|u\|_{L^{p^*}(U)} \leq C(p,n,U)\|u\|_{W^{1,p}(U)},\]
where $C(p,n,U) > 0$ is a constant.
\end{theorem}
\begin{proof}
In progress...
\end{proof}

\begin{theorem}[Estimates for $W_{0}^{1,p}$, where $1 \leq p < n$]
	Suppose $u \in W_{0}^{1,p}(U)$ for some $1 \leq p < n$. Then we get the estimate
	\[\|u\|_{L^{q}(U)} \leq C(p,q,n,U)\|Du\|_{L^{p}(U)},\]
	for all $1 \leq q \leq p^*$, where $C(p,q,n,U) > 0$ is a constant.
\end{theorem}
\begin{proof}
In progress...
\end{proof}

\begin{remark}
If $U$ is bounded as we supposed, then on $W_{0}^{1,p}$, the norm $\|Du\|_{L^{p}} \equiv \|u\|_{W^{1,p}(U)}$,
meaning this is still an estimate for functions in Sobolev spaces.
\end{remark}

\begin{theorem}[Morrey's inequality]
Suppose $n < p \leq \infty$. Then there exists a constant $C(p,n) > 0$ such that
	\[\|u\|_{C^{0,\gamma}(\mathbb{R}^n)} \leq C(p,n)\|u\|_{W^{1,p}(\mathbb{R}^n)},\]
for all $u \in W^{1,p}(U)$ with $u \in C^1(\mathbb{R}^n)$, where $\gamma = 1 - \frac{n}{p}$.
\end{theorem}
\begin{proof}
In progress...
\end{proof}

\begin{theorem}[Estimates for $W^{1,p}$, where $n < p \leq \infty$]
Suppose $\partial U$ is $C^1$,
	and suppose $u \in W^{1,p}(U)$ for some $n < p \leq \infty$. It then follows that a continuous version $u^* \in C^{0,\gamma}(\overline{U})$,
where $\gamma = 1 - \frac{n}{p}$, with the estimate that
	\[\|u^*\|_{C^{0,\gamma}(\overline{U})} \leq C(p,n,U)\|u\|_{W^{1,p}(U)},\]
where $C(p,n,U) > 0$ is a constant.
\end{theorem}
\begin{proof}
In progress...
\end{proof}

\newpage

\section{Notation}
\begin{enumerate}[(i)]
	\item A \textit{multiindex} is a vector $\alpha = (\alpha_1, \dots, \alpha_n)$ where each component $\alpha_i \in \mathbb{N}_0$.
A multiindex has an order defined by
\[|\alpha| = \alpha_1 + \dots + \alpha_n.\]

\item Using our definition of a multiindex and letting $u(x)$ be some function, we define
\[D^{\alpha}u(x) = \frac{\partial^{|\alpha|}u(x)}{\partial x_1^{\alpha_1} \dots \partial x_n^{\alpha_n}} = \partial_{x_1}^{\alpha_1}\dots\partial_{x_n}^{\alpha_n}u(x).\]

\item Let $U,V \subset \mathbb{R}^n$. Then define
\[V \subset\subset U\]
to be when $V \subset \overline{V} \subset U$ and $\overline{V}$ is compact. In plain english this means $V$ is \textit{compactly contained} in $U$.

\item Let $f$ and $g$ be functions. Then define $*$ to be the \textit{convolution operator} where
\[(f*g)(x) = \int_{-\infty}^{\infty}{f(\tau)g(x - \tau)d\tau} = \int_{-\infty}^{\infty}{f(x - \tau)g(\tau)d\tau}\]
is the \textit{convolution} of the functions $f$ and $g$ which results in a third function that expresses
how one of the functions modifies the other. Note that I am assuming $f$ and $g$ are both supported on an
infinite interval, which may not always be the case.

\item Let $u(x)$ be a function where $x \in \mathbb{R}^n$. Then the \textit{gradient vector} of $u$ is 
	\[Du(x_1,\dots,x_n) = (u_{x_1},...,u_{x_n}).\]
\end{enumerate}


\end{document}
