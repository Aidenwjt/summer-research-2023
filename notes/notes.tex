\documentclass[11pt]{article}
%\setlength{\textwidth}{430pt}\setlength{\oddsidemargin}{11pt}
\usepackage{amssymb}
\usepackage{amsthm}
\usepackage{amsmath}
\usepackage{enumerate}
\usepackage{fancyhdr}
\usepackage[bottom=0.9in,top=0.9in]{geometry}

\theoremstyle{definition}
\newtheorem{theorem}{Theorem}[section]
\newtheorem{definition}[theorem]{Definition}
\newtheorem*{remark}{Remark}
\newtheorem*{motivation}{Motivation}

\begin{document}
\pagestyle{fancy}
\fancyhead{}
\fancyhead[C]{\textbf{2023 Summer Research Notes}}
\tableofcontents
\newpage
\section{Prior Concepts to Know}
\subsection{$L^p$ Spaces}
\subsection{Hilbert Spaces}
\subsection{Linear Spaces}
\subsection{Normed Linear Spaces}
\subsection{Metric Spaces}
\subsection{Banach Spaces}
\subsection{Weak Derivatives}
\subsection{Mollifiers}
\subsection{Partition of Unity}

\newpage

\section{Hölder Spaces}
\subsection{Hölder Continuous Functions}
\subsection{Hölder Spaces are Banach Spaces}

\newpage

\section{Sobolev Spaces}
\subsection{Sobolev Inequalities}
\begin{motivation}
The main goal of Sobolev inequalities is to find embeddings of Sobolev spaces in other spaces that may be easier to work with, more desirable,
or simply imply nice properties. Following Evans we will first consider functions in the space $W^{1,p}$, which gives rise to three cases being
$1 \leq p < n$, $p=n$, or $n < p \leq \infty$.
\end{motivation}


\newpage

\section{Notation}
\begin{enumerate}[(i)]
\item A multiindex is a vector $\alpha = (\alpha_1, \dots, \alpha_n)$ where each component $\alpha_i \in \mathbb{N}_0$.
A multiindex has an order defined by
\[|\alpha| = \alpha_1 + \dots + \alpha_n.\]

\item Using our definition of a multiindex and letting $u(x)$ be some function, we define
\[D^{\alpha}u(x) = \frac{\partial^{|\alpha|}u(x)}{\partial x_1^{\alpha_1} \dots \partial x_n^{\alpha_n}} = \partial_{x_1}^{\alpha_1}\dots\partial_{x_n}^{\alpha_n}u(x).\]

\item Let $U,V \subset \mathbb{R}^n$. Then define
\[V \subset\subset U\]
to be when $V \subset \overline{V} \subset U$ and $\overline{V}$ is compact. In plain english this means $V$ is \textit{compactly conatined} in $U$.

\item Let $f$ and $g$ be functions. Then define $*$ to be the Convolution operator where
\[(f*g)(x) = \int_{-\infty}^{\infty}{f(\tau)g(x - \tau)d\tau} = \int_{-\infty}^{\infty}{f(x - \tau)g(\tau)d\tau}\]
is the Convolution of the functions $f$ and $g$ which results in a third function that expresses
how one of the functions modifies the other. Note that I am assuming $f$ and $g$ are both supported on an
infinite interval, which may not always be the case.
\end{enumerate}


\end{document}
