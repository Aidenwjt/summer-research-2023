\documentclass[11pt]{article}
%\setlength{\textwidth}{430pt}\setlength{\oddsidemargin}{11pt}
\usepackage{amssymb}
\usepackage{amsthm}
\usepackage{amsmath}
\usepackage{enumerate}
\usepackage{fancyhdr}
\usepackage{hyperref}
\hypersetup{colorlinks=true,linktoc=all,linkcolor=blue}
\usepackage[bottom=0.9in,top=0.9in]{geometry}

\theoremstyle{definition}
\newtheorem*{theorem}{Theorem}
\newtheorem*{definition}{Definition}
\newtheorem*{remark}{Remark}
\newtheorem*{motivation}{Motivation}

\begin{document}
\pagestyle{fancy}
\fancyhead{}
\fancyhead[C]{\textbf{2023 Summer Research Notes}}
\tableofcontents
\newpage
\section{Prior Concepts to Know}
\subsection{$L^p$ Spaces}
\subsection{Hilbert Spaces}
\subsection{Linear Spaces}
\subsection{Normed Linear Spaces}
\subsection{Metric Spaces}
\subsection{Banach Spaces}
\subsection{Weak Derivatives}
\subsection{Mollifiers}
\subsection{Partition of Unity}

\newpage

\section{Hölder Spaces}
\subsection{Hölder Continuous Functions}
\subsection{Hölder Spaces are Banach Spaces}

\newpage

\section{Sobolev Spaces}
\subsection{Sobolev Inequalities}
\begin{motivation}
The main goal of Sobolev inequalities is to find embeddings of Sobolev spaces in other spaces that may be easier to work with, more desirable,
or simply imply nice properties. These inequalities essentially provide estimates for functions in Sobolev spaces using other
well know function spaces. Many of the following proofs are taken from Evans.
\end{motivation}
\begin{theorem}[Gagliardo-Nirenberg-Sobolev Inequality]
Suppose $1\leq p < n$. There exists a constant $C > 0$, depending only on $p$ and $n$, such that
	\[\|u\|_{L^{p^*}(\mathbb{R}^n)} \leq C\|Du\|_{L^{p}(\mathbb{R}^n)},\]
for all $u \in C_{c}^{1}(\mathbb{R}^n)$.
\end{theorem}
\begin{remark}
This Theorem proves the inequalitiy
	\[\|u\|_{L^{q}(\mathbb{R}^n)} \leq C\|Du\|_{L^{p}(\mathbb{R}^n)},\]
which can only be possible if $q=p*$ ($p*$ being the Sobolev conjugate of $p$).
\end{remark}

\begin{theorem}[Estimates for $W^{1,p}$, where $1 \leq p < n$]
Let $U \subset \mathbb{R}^n$ be a bounded open set, and suppose $\partial U$ is $C^1$.
Suppose $u \in W^{1,p}(U)$ for some $1 \leq p < n$. It then follows that $u \in L^{p^*}$, with the estimate that
	\[\|u\|_{L^{p^*}(U)} \leq C\|u\|_{W^{1,p}(U)},\]
where $C > 0$ is constant and only depending on $p$, $n$, and $U$.
\end{theorem}
\begin{remark}
This Theorem provides a nice estimate for functions in Sobolev spaces where $u$ has weak derivatives of order $k = 1$,
but where $p$ is less than the dimension of $U \subset \mathbb{R}^n$.
\end{remark}

\begin{theorem}[Estimates for $W_{0}^{1,p}$, where $1 \leq p < n$]
Let $U \subset \mathbb{R}^n$ be a bounded open set.
	Suppose $u \in W_{0}^{1,p}(U)$ for some $1 \leq p < n$. Then we get the estimate
	\[\|u\|_{L^{q}(U)} \leq C\|Du\|_{L^{p}(U)},\]
for all $1 \leq q \leq p^*$, where $C > 0$ is constant and only depending on $p$, $q$, $n$, and $U$.
\end{theorem}
\begin{remark}
This Theorem gives a nice estimate for the closure of $C_{c}^{\infty}(U)$ in $W^{1,p}(U)$,
where derivatives of functions of certain orders in $W^{1,p}(U)$ evaluate to zero on $\partial U$.
Also, if $U$ is bounded as we supposed, then on $W_{0}^{1,p}$ the norm $\|Du\|_{L^{p}} \equiv \|u\|_{W^{1,p}(U)}$,
meaning this is still an estimate for functions in Sobolev spaces.
\end{remark}

\begin{theorem}[Morrey's Inequality]
Suppose $n < p \leq \infty$. Then there exists a constant $C > 0$, depending only on $p$ and $n$, such that
	\[\|u\|_{C^{0,\gamma}(\mathbb{R}^n)} \leq C\|u\|_{W^{1,p}(\mathbb{R}^n)},\]
for all $u \in W^{1,p}(U)$ with $u \in C^1(\mathbb{R}^n)$, where $\gamma = 1 - \frac{n}{p}$.
\end{theorem}
\begin{remark}
Since $n < p \leq \infty$ and $u \in W^{1,p}(U)$ it follows that $u$ is H\"{o}lder continuous allowing us
	to use the H\"{o}lder norm to get a nice estimate.
\end{remark}

\begin{theorem}[Estimates for $W^{1,p}$, where $n < p \leq \infty$]
Let $U \subset \mathbb{R}^n$ be a bounded open set, and suppose $\partial U$ is $C^1$.
	Suppose $u \in W^{1,p}(U)$ for some $n < p \leq \infty$. It then follows that a continuous version $u^* \in C^{0,\gamma}(\overline{U})$,
where $\gamma = 1 - \frac{n}{p}$, with the estimate that
	\[\|u^*\|_{C^{0,\gamma}(\overline{U})} \leq C\|u\|_{W^{1,p}(U)},\]
where $C > 0$ is constant and only depending on $p$, $n$, and $U$.
\end{theorem}
\begin{remark}
	Again, this Theorem provides a nice estimate using the H\"{o}lder norm and a continuous version of the said function,
	essentially allowing us to estimate the function in the Sobolev space with continuous functions.
\end{remark}

\newpage

\section{Notation}
\begin{enumerate}[(i)]
\item A multiindex is a vector $\alpha = (\alpha_1, \dots, \alpha_n)$ where each component $\alpha_i \in \mathbb{N}_0$.
A multiindex has an order defined by
\[|\alpha| = \alpha_1 + \dots + \alpha_n.\]

\item Using our definition of a multiindex and letting $u(x)$ be some function, we define
\[D^{\alpha}u(x) = \frac{\partial^{|\alpha|}u(x)}{\partial x_1^{\alpha_1} \dots \partial x_n^{\alpha_n}} = \partial_{x_1}^{\alpha_1}\dots\partial_{x_n}^{\alpha_n}u(x).\]

\item Let $U,V \subset \mathbb{R}^n$. Then define
\[V \subset\subset U\]
to be when $V \subset \overline{V} \subset U$ and $\overline{V}$ is compact. In plain english this means $V$ is \textit{compactly conatined} in $U$.

\item Let $f$ and $g$ be functions. Then define $*$ to be the Convolution operator where
\[(f*g)(x) = \int_{-\infty}^{\infty}{f(\tau)g(x - \tau)d\tau} = \int_{-\infty}^{\infty}{f(x - \tau)g(\tau)d\tau}\]
is the Convolution of the functions $f$ and $g$ which results in a third function that expresses
how one of the functions modifies the other. Note that I am assuming $f$ and $g$ are both supported on an
infinite interval, which may not always be the case.
\end{enumerate}


\end{document}
